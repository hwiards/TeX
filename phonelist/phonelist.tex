%% Phonelist
%% Autor: Hilko Wiards
\documentclass{article}
\usepackage{xparse}


%% Define the Domain here!
\newcommand{\domain}{dummymail.com}


%% Commands for easy useage.
\newcommand{\teleEntry}[3]{#1 & #2 & #3 \\}
\NewDocumentCommand{\teleMailEntry}{m m m g}{
	\IfValueTF{#4}{#1 & #2 & #3 & #4\\}{#1 & #2 & \domain & #3\\}
}

\begin{document}
	\centering	
	\pagestyle{empty}	
		
	%%% Enter the Title here. (Organisation/Groupname etc.)
	\huge Phonelist - Title
	\vspace{10pt}
	
	%%% Enter the Subtitle here. (Telephonelist etc.)
	\large -- Phonelist - Subtitle --
	\normalsize
	
	
	%%% Choose one of the following options.
	%%% The second one aligns emails at the @ symbol.
	
	%% The usage of \teleEntry{Name}{Email-Adress}{Phone Nr.}
	\vspace{30pt}
	\begin{tabular}{lll}
		
		%% Enter contacts here.
		\teleEntry{John Doe}{John.Doe@dummymail.com}{0125 898454783}
		\teleEntry{Johanna Doe}{Johanna.Doe@dummymail.com}{0125 898454783}
		\teleEntry{Kim Doe}{Kim@short.doe}{0125 898454783}

	\end{tabular}


	%% The usage of \teleMailEntry{Name}{Mail local-part}{Mail domain}{Phone Nr.}
	%% The Domain is optional 
	%% You can specify a default domain at the top of this document. 
	\vspace{30pt}
	\begin{tabular}{lr@{@}ll}
		% Enter contacts here.
		\teleMailEntry{John Doe}{John.Doe}{dummymail.com}{0125 898454783}
		\teleMailEntry{Johanna Doe}{Johanna.Doe}{0125 898454783} %% Domain ommited.
		\teleMailEntry{Kim Doe}{Kim}{short.doe}{0125 898454783}
	\end{tabular}

\end{document}

